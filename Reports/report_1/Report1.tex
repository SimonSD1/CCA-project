\documentclass[a4paper,12pt]{article}

\usepackage[utf8]{inputenc}
\usepackage[T1]{fontenc}
\usepackage{lmodern}
\usepackage{amsmath} % For math environments
\usepackage{amssymb} % For math symbols
\usepackage{hyperref} % For hyperlinks

\title{Report 1}
\author{Simon Sepiol-Duchemi Joshua Setia}
\date{\today}

\begin{document}

\maketitle

\section{Proof of Descarte's rule of sign}
Rewriting of the proof from Wikipedia.\footnote{For reference, see the original proof on Wikipedia: \href{https://en.wikipedia.org/wiki/Descartes\%27_rule_of_signs}{Descartes' Rule of Signs on Wikipedia}.}


\subsection*{Preliminary Definitions}
\begin{itemize}
    \item Write the polynomial \( f(x) \) as 
    \[
    f(x) = \sum_{i=0}^{n} a_i x^{b_i},
    \]
    where we have integer powers \( 0 \leq b_0 < b_1 < \cdots < b_n \), and nonzero coefficients \( a_i \neq 0 \).
    If \( b_0 > 0 \), then we can divide the polynomial by \( x^{b_0} \), which would not change its number of strictly positive roots. Thus, without loss of generality (WLOG), let \( b_0 = 0 \).

    \item Let \( V(f) \) be the number of sign changes of the coefficients of \( f \), meaning the number of \( k \) such that \( a_k a_{k+1} < 0 \).
    \item Let \( Z(f) \) be the number of strictly positive roots (counting multiplicity).

\end{itemize}

\subsection*{Theorem}
The number of strictly positive roots (counting multiplicity) of \( f \) is equal to the number of sign changes in the coefficients of \( f \), minus a nonnegative even number.


\subsection*{Lemma}
\begin{itemize}
    \item If \( a_n a_0 > 0 \), then \( Z(f) \) is even.
    \item If \( a_n a_0 < 0 \), then \( Z(f) \) is odd.
\end{itemize}

\subsection*{Proof of Lemma}
If \(a_n a_0 > 0\), then both are negative or both are positive, so for example if both 
are positive the function start at \(a_0 >0\) and end at \(f(+\infty)=+\infty\), so it must
cross the \(x\)-axis an even number of time. So \( Z(f) \) is even. The case of \(a_n a_0 > 0\)\(a_n a_0 > 0\) is similar.

\subsection*{Proof of the main theorem}

Firstly, if \(a_n a_0 >0\), then there is an even number of sign change, and if \(a_n a_0<0\), there is an odd number of sign change.
So, the parity of \(Z(f)\) is the same as \(V(f)\).

We need to show that \(Z(f) < V(f)\), we proceed by inducting on n the degree of \(f\).
\\

For \(n=1\), \(f=a_1x^{b_1}+a_0\), there is always one and only one real root at 
\(x=\frac{a_1}{-a_0}\). So there is one positive real root only if \(a_0\) and \(a_1\) have a different sign, \(Z(f) \leq V(f)\).
\\

Now assume \(n \geq 2\).
\\

By using the induction hypothesis, \(Z(f') = V(f') - 2s\) for some integer \(s \geq 2\).

By the Rolles's theorem, there exist at least one positive root of \(f'\) between any two different positive roots of f.

Any k-multiple positive root of \(f\) is a \(k-1\) multiple root of \(f'\). Thus \(Z(f') \geq Z(f)-1\).

If \(a_0 a_1 > 0\), then \(V(f')=V(f)\), else \(V(f')=V(f)-1\). In both case, \(V(f')=V(f)\).

Together, we have
\[Z(f) \leq Z(f')+1 = V(f') -2s +1 \leq V(f) -2s+1 \leq V(f+1)\]
Further, since \(Z(f)\) and \(V(f)\) have the same parity, we have \(Z(f) \leq V(f)\).

\end{document}
